\documentclass[12pt]{article}
\usepackage{pmmeta}
\pmcanonicalname{TimedependentExampleOfHeatEquation}
\pmcreated{2013-03-22 16:32:47}
\pmmodified{2013-03-22 16:32:47}
\pmowner{pahio}{2872}
\pmmodifier{pahio}{2872}
\pmtitle{time-dependent example of heat equation}
\pmrecord{10}{38729}
\pmprivacy{1}
\pmauthor{pahio}{2872}
\pmtype{Example}
\pmcomment{trigger rebuild}
\pmclassification{msc}{35Q99}
\pmsynonym{solving a time-dependent heat equation}{TimedependentExampleOfHeatEquation}
%\pmkeywords{heat equation}
%\pmkeywords{Fourier double series}
%\pmkeywords{eigenvalues}
\pmrelated{ExampleOfSolvingTheHeatEquation}
\pmrelated{EigenvalueProblem}

\endmetadata

% this is the default PlanetMath preamble.  as your knowledge
% of TeX increases, you will probably want to edit this, but
% it should be fine as is for beginners.

% almost certainly you want these
\usepackage{amssymb}
\usepackage{amsmath}
\usepackage{amsfonts}

% used for TeXing text within eps files
%\usepackage{psfrag}
% need this for including graphics (\includegraphics)
%\usepackage{graphicx}
% for neatly defining theorems and propositions
 \usepackage{amsthm}
% making logically defined graphics
%%%\usepackage{xypic}

% there are many more packages, add them here as you need them

% define commands here

\theoremstyle{definition}
\newtheorem*{thmplain}{Theorem}

\begin{document}
The initial temperature (at\, $t = 0$) of a \PMlinkescapetext{thin homogeneous} plate
   $$A = \{(x,\,y)\in\mathbb{R}^2\vdots\,\,\,0 < x < a,\,\,0 < y < b\}$$
in the $xy$-plane is given by the function\, $f = f(x,\,y)$.\, The faces of the plate are supposed completely isolating.\, After the \PMlinkescapetext{moment}\, $t = 0$\, the boundaries of $A$ are held in the temperature $0$.\, Determine the temperature function
                         $$u = u(x,\,y,\,t)$$
on $A$ (where $t$ is the time).

Since it's a question of a two-dimensional heat \PMlinkescapetext{flow}, the heat equation gets the form
\begin{align}
  \nabla^2u \equiv u''_{xx}+u''_{yy} = \frac{1}{c^2}u'_t.
\end{align}
One have to find for (1) a solution function $u$ which satisfies the initial condition
\begin{align}
  u(x,\,y,\,0) = f(x,\,y)\,\,\,\mathrm{in}\,\,\,A
\end{align}
and the boundary condition
\begin{align}
  u(x,\,y,\,t) = 0\,\,\,     \mathrm{on\,\,boundary\,\,of}\,\,A\,\,\,\mathrm{for}\,\,t > 0.
\end{align}

For finding a simple solution of the differential equation (1) we try the form
\begin{align}
u(x,\,y,\,t) := X(x)Y(y)T(t),
\end{align}
whence the boundary condition reads
\begin{align}
X(0) = Y(0) = X(a) = Y(b) = 0.
\end{align}
Substituting (4) in (1) and dividing this equation by $XYT$ give the form
\begin{align}
\frac{X''}{X}+\frac{Y''}{Y} = \frac{1}{c^2}\!\cdot\!\frac{T'}{T}.
\end{align}
It's easily understood that such a condition requires that the both addends of the left side and the right side ought to be constants:
\begin{align}
\frac{X''}{X} = -k_1^2,\,\, \frac{Y''}{Y} = -k_2^2,\,\, 
\frac{1}{c^2}\!\cdot\!\frac{T'}{T} = -k^2,
\end{align}
where\, $k^2 = k_1^2+k_2^2$.\, We soon explain why these constants are negative.
Because the equations (7) may be written
$$X'' = -k_1^2X,\,\,\, Y'' = -k_2^2Y,\,\,\, T' = -k^2c^2T,$$ 
the general solutions of these ordinary differential equations are
\begin{align}
\begin{cases}
 X = C_1\cos{k_1x}+D_1\sin{k_1x},\\
 Y = C_2\cos{k_2y}+D_2\sin{k_2y},\\
 T = Ce^{-k^2c^2t}.
\end{cases}
\end{align}
Now we remark that if the right side of the third equation (7) were\, $+k^2$, then we had\, $T = Ce^{k^2c^2t}$\, which is impossible, since such a $T$ and along with this also the temperature\, $u = XYT$\, would ascend infinitely when\, $t\to\infty$.\, And since, by symmetry, the right sides the two first equations (7) must have the same sign, also they must by (6) be negative.

The two first boundary conditions (5) imply by (8) that\, $C_1 = C_2 = 0$,\, and then the two last conditions (5) require that
$$D_1\sin{k_1a} = 0,\,\,\, D_2\sin{k_2b} = 0.$$
If we had\, $D_1 = 0$\, or\, $D_2 = 0$,\, then $X$ or $Y$ would vanish identically, which cannot occur.\, Thus we have
$$\sin{k_1a} = 0\,\,\, \mathrm{and}\,\,\, \sin{k_2b} = 0,$$
whence only the eigenvalues
\begin{align*}
\begin{cases}
  k_1 = \frac{m\pi}{a}\,\,\,(m = 1,\,2,\,3,\,\ldots)\\
  k_2 = \frac{n\pi}{b}\,\,\,(n = 1,\,2,\,3,\,\ldots)
\end{cases}
\end{align*}
are possible for the obtained $X$ and $Y$.\, Considering the equation\, $k^2 = k_1^2+k_2^2$\, we may denote
\begin{align}
 q_{mn} := k^2c^2 =
\left[\left(\frac{m\pi}{a}\right)^2\!+\!\left(\frac{n\pi}{b}\right)^2\right]c^2
\end{align}
 for all\,\, $m,\,n \in \mathbb{Z}_+.$

Altogether we have infinitely many solutions
$$u_{mn} = XYT = CD_1D_2e^{-q_{mn}t}\sin\frac{m\pi x}{a}\sin\frac{n\pi y}{b}
= c_{mn}e^{-q_{mn}t}\sin\frac{m\pi x}{a}\sin\frac{n\pi y}{b}$$
of the equation (1), where the coefficients $c_{mn}$ are, for the present, arbitrary constants.\, These solutions fulfil the boundary condition (3).\, The sum of the solutions, i.e. the double series
\begin{align}
  u(x,\,y,\,t) := \sum_{m=1}^\infty\sum_{n=1}^\infty 
  c_{mn}e^{-q_{mn}t}\sin\frac{m\pi x}{a}\sin\frac{n\pi y}{b},
\end{align}
provided it converges, is also a solution of the linear differential equation (1) and fulfils the boundary condition.\, In order to fulfil also the initial condition (2), one must have
$$\sum_{m=1}^\infty\sum_{n=1}^\infty 
c_{mn}e^{-q_{mn}t}\sin\frac{m\pi x}{a}\sin\frac{n\pi y}{b} = f(x,\,y).$$
But this equation presents the Fourier double sine series \PMlinkescapetext{expansion} of\, $f(x,\,y)$\, in the rectangle $A$, and therefore we have the expression
\begin{align}  
c_{mn} := \frac{4}{ab}
\int_0^a\!\int_0^b f(x,\,y)\,\sin\frac{m\pi x}{a}\sin\frac{n\pi y}{b}\,dx\,dy
\end{align}
for the coefficients.

The result of calculating the solution of our problem is the temperature function (10) with the formulae (9) and (11).

\begin{thebibliography}{9}
\bibitem{K.V.}{\sc K. V\"ais\"al\"a:} {\em Matematiikka IV}.\, Hand-out Nr. 141.\quad Teknillisen korkeakoulun ylioppilaskunta, Otaniemi, Finland (1967).
\end{thebibliography}
 



%%%%%
%%%%%
\end{document}
