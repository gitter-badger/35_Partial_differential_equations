\documentclass[12pt]{article}
\usepackage{pmmeta}
\pmcanonicalname{SphericalMean}
\pmcreated{2013-03-22 14:09:04}
\pmmodified{2013-03-22 14:09:04}
\pmowner{rspuzio}{6075}
\pmmodifier{rspuzio}{6075}
\pmtitle{spherical mean}
\pmrecord{7}{35568}
\pmprivacy{1}
\pmauthor{rspuzio}{6075}
\pmtype{Definition}
\pmcomment{trigger rebuild}
\pmclassification{msc}{35L05}
\pmclassification{msc}{26E60}
\pmrelated{WaveEquation}

% this is the default PlanetMath preamble.  as your knowledge
% of TeX increases, you will probably want to edit this, but
% it should be fine as is for beginners.

% almost certainly you want these
\usepackage{amssymb}
\usepackage{amsmath}
\usepackage{amsfonts}

% used for TeXing text within eps files
%\usepackage{psfrag}
% need this for including graphics (\includegraphics)
%\usepackage{graphicx}
% for neatly defining theorems and propositions
%\usepackage{amsthm}
% making logically defined graphics
%%%\usepackage{xypic}

% there are many more packages, add them here as you need them

% define commands here
\def\sse{\subseteq}
\def\bigtimes{\mathop{\mbox{\Huge $\times$}}}
\def\impl{\Rightarrow}
\def\R{\mathbb{R}}
\def\del{\partial}
\begin{document}
\PMlinkescapeword{name}
\PMlinkescapeword{identity}
\PMlinkescapeword{independent}
\PMlinkescapeword{unit}

Let $h$ be a function (usually real or complex valued) on $\R^n$ ($n\ge1$).
Its \emph{spherical mean} at point $x$ over a sphere of radius $r$ is defined as
\begin{equation*}
  M_h(x,r) = \frac{1}{A(n-1)} \int_{\|\xi\|=1} h(x+r\xi)\, dS
    = \frac{1}{A(n-1,r)} \int_{\|\xi\|=|r|} h(x+\xi)\, dS,
\end{equation*}
where the integral is over the surface of the unit $n-1$-sphere. Here $A(n-1)$ is is the area of the unit sphere, while $A(n-1,r)=r^{n-1}A(n-1)$ is the \PMlinkname{area of a sphere of radius $r$}{AreaOfTheNSphere}. In essense, the spherical mean $M_h(x,r)$ is just the average of $h$ over the surface of a sphere of radius $r$ centered at $x$, as the name suggests.

The spherical mean is defined for both positive and negative $r$ and is
independent of its sign. As $r\to 0$, if $h$ is continuous, $M_h(x,r)\to
h(x)$. If $h$ has two continuous derivatives (is in $C^2(\R^n)$) then the
following identity holds:
\begin{equation*}
  \nabla^2_x M_h(x,r) = \left(\frac{\del^2}{\del r^2} + \frac{n-1}{r}
    \frac{\del}{\del r}\right) M_h(x,r),
\end{equation*}
where $\nabla^2$ is the Laplacian.

Spherical means are used to obtain an explicit general solution for the wave
equation in $n$ space and one time dimensions.
%%%%%
%%%%%
\end{document}
