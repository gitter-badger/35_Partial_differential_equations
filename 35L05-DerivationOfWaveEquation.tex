\documentclass[12pt]{article}
\usepackage{pmmeta}
\pmcanonicalname{DerivationOfWaveEquation}
\pmcreated{2013-03-22 18:46:36}
\pmmodified{2013-03-22 18:46:36}
\pmowner{pahio}{2872}
\pmmodifier{pahio}{2872}
\pmtitle{derivation of wave equation}
\pmrecord{10}{41570}
\pmprivacy{1}
\pmauthor{pahio}{2872}
\pmtype{Derivation}
\pmcomment{trigger rebuild}
\pmclassification{msc}{35L05}
\pmrelated{Slope}
\pmrelated{MeanValueTheorem}
\pmrelated{DerivationOfHeatEquation}

% this is the default PlanetMath preamble.  as your knowledge
% of TeX increases, you will probably want to edit this, but
% it should be fine as is for beginners.

% almost certainly you want these
\usepackage{amssymb}
\usepackage{amsmath}
\usepackage{amsfonts}

% used for TeXing text within eps files
%\usepackage{psfrag}
% need this for including graphics (\includegraphics)
%\usepackage{graphicx}
% for neatly defining theorems and propositions
 \usepackage{amsthm}
% making logically defined graphics
%%%\usepackage{xypic}

% there are many more packages, add them here as you need them

% define commands here

\theoremstyle{definition}
\newtheorem*{thmplain}{Theorem}

\begin{document}
\PMlinkescapeword{string} \PMlinkescapeword{constant} \PMlinkescapeword{mass}
\PMlinkescapeword{force}
Let a string of \PMlinkescapetext{homogeneous} matter be tightened between the points \,$x = 0$\, and\, $x = p$\, of the $x$-axis and let the string be made vibrate in the $xy$-plane.\, Let the \PMlinkescapetext{line density of mass} of the string be the constant $\sigma$.\, We suppose that the amplitude of the vibration is so small that the tension $\vec{T}$ of the string can be regarded to be constant.

The position of the string may be represented as a function
$$y \;=\; y(x,\,t)$$
where $t$ is the time.\, We consider an element $dm$ of the string situated on a tiny interval \, $[x,\,x\!+\!dx]$;\, thus its mass is $\sigma\,dx$.\, If the angles the vector $\vec{T}$ at the ends $x$ and $x\!+\!dx$ of the element forms with the direction of the $x$-axis are $\alpha$ and $\beta$, then the scalar \PMlinkescapetext{components of the resultant} force $\vec{F}$ of all \PMlinkescapetext{forces} on $dm$ (the gravitation omitted) are
$$F_x \;=\; -T\cos\alpha+T\cos\beta, \quad F_y \;=\; -T\sin\alpha+T\sin\beta.$$
Since the angles $\alpha$ and $\beta$ are very small, the ratio
$$\frac{F_x}{F_y} \;=\; \frac{\cos\beta-\cos\alpha}{\sin\beta-\sin\alpha} \;=\; 
\frac{-2\sin\frac{\beta-\alpha}{2}\sin\frac{\beta+\alpha}{2}}{2\sin\frac{\beta-\alpha}{2}\cos\frac{\beta+\alpha}{2}},$$
having the expression \,$-\tan\frac{\beta+\alpha}{2}$, also is very small.\, Therefore we can omit the horizontal component $F_x$ and think that the vibration of all elements is strictly vertical.\, Because of the smallness of the angles $\alpha$ and $\beta$, their sines in the expression of 
$F_y$ may be replaced with their tangents, and accordingly
$$F_y \;=\; T\cdot(\tan\beta-\tan\alpha) \;=\; T\,[y'_x(x\!+\!dx,\,t)-y'_x(x,\,t)] \;=\; T\,y''_{xx}(x,\,t)\,dx,$$
the last form due to the mean-value theorem.

On the other hand, by Newton the force equals the mass times the acceleration:
$$F_y \;=\; \sigma\,dx\,y''_{tt}(x,\,t)$$
Equating both expressions, dividing by $T\,dx$ and denoting\, $\displaystyle\sqrt{\frac{T}{\sigma}} = c$,\, we obtain the partial differential equation
\begin{align}
y''_{xx} \;=\; \frac{1}{c^2}y''_{tt}
\end{align}
for the equation of the \PMlinkescapetext{transversely} vibrating string.\\

But the equation (1) don't suffice to entirely determine the vibration.\, Since the end of the string are immovable,the function\, $y(x,\,t)$\, has in \PMlinkescapetext{addition} to satisfy the boundary conditions
\begin{align}
y(0,\,t) \;=\; y(p,\,t) \;=\; 0
\end{align}
The vibration becomes completely determined when we know still e.g. at the beginning\, $t = 0$\, the position $f(x)$ of the string and the initial velocity $g(x)$ of the points of the string; so there should be the initial conditions
\begin{align}
y(x,\,0) \;=\; f(x), \quad y'_t(x,\,0) \;=\; g(x).
\end{align}


The equation (1) is a special case of the general wave equation
\begin{align}
\nabla^2u \;=\; \frac{1}{c^2}u''_{tt}
\end{align}
where\, $u =u(x,\,y,\,z,\,t)$.\, The equation (4) rules the spatial waves in $\mathbb{R}$.\, The number $c$ can be shown to be the velocity of propagation of the wave motion.

\begin{thebibliography}{9}
\bibitem{K.V.}{\sc K. V\"ais\"al\"a:} {\em Matematiikka IV}.\, Handout Nr. 141.\quad Teknillisen korkeakoulun ylioppilaskunta, Otaniemi, Finland (1967).
\end{thebibliography}
%%%%%
%%%%%
\end{document}
