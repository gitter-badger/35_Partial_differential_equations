\documentclass[12pt]{article}
\usepackage{pmmeta}
\pmcanonicalname{DiracEquation}
\pmcreated{2013-03-22 17:54:46}
\pmmodified{2013-03-22 17:54:46}
\pmowner{Raphanus}{20453}
\pmmodifier{Raphanus}{20453}
\pmtitle{Dirac equation}
\pmrecord{21}{40407}
\pmprivacy{1}
\pmauthor{Raphanus}{20453}
\pmtype{Definition}
\pmcomment{trigger rebuild}
\pmclassification{msc}{35Q40}
\pmclassification{msc}{81Q05}
%\pmkeywords{relativistic}
%\pmkeywords{D'Alembertian}
\pmrelated{Spinor}
\pmrelated{KleinGordonEquation}
\pmrelated{SchrodingersWaveEquation}
\pmrelated{PauliMatrices}
\pmdefines{Feynman slash notation}
\pmdefines{Dirac matrices}

\endmetadata

% this is the default PlanetMath preamble.  as your knowledge
% of TeX increases, you will probably want to edit this, but
% it should be fine as is for beginners.

% almost certainly you want these
\usepackage{amssymb}
\usepackage{amsmath}
\usepackage{amsfonts}

% used for TeXing text within eps files
%\usepackage{psfrag}
% need this for including graphics (\includegraphics)
%\usepackage{graphicx}
% for neatly defining theorems and propositions
%\usepackage{amsthm}
% making logically defined graphics
%%%\usepackage{xypic}

% there are many more packages, add them here as you need them
\usepackage{cancel}
% define commands here

\begin{document}
The Dirac equation is an equation derived by Paul Dirac in 1927 that describes relativistic spin $1/2$ particles (fermions). It is given by:
\[
(\gamma^\mu \partial_\mu - im)\psi = 0
\]
The Einstein summation convention is used.
\subsection{Derivation}
Mathematically, it is interesting as one of the first uses of the spinor calculus in mathematical physics. Dirac began with the relativistic equation of total energy:
\[
E = \sqrt{p^2c^2 + m^2c^4}
\]
As Schr\"odinger had done before him, Dirac then replaced $p$ with its quantum mechanical operator, $\hat{p} \Rightarrow i\hbar \nabla$. Since he was looking for a Lorentz-invariant equation, he replaced $\nabla$ with the D'Alembertian or wave operator
\[
\Box = \nabla^2 - \frac{1}{c^2} \frac{\partial^2}{\partial t^2}
\]
Note that some authors use $\Box^2$ for the D'Alembertian. Dirac was now faced with the problem of how to take the square root of an expression containing a differential operator. He proceeded to factorise the d'Alembertian as follows:
\[
\nabla^2 - \frac{1}{c^2} \frac{\partial^2}{\partial t^2} = (a^0 \frac{\partial}{\partial x} + a^1 \frac{\partial}{\partial y} + a^2 \frac{\partial}{\partial z} + a^3\frac{i}{c} \frac{\partial}{\partial t})^2
\]
Multiplying this out, we find that:
\[
(a^0)^2 = (a^1)^2 = (a^2)^2 = (a^3)^2 = 1
\]
And
\[
a^0a^1 + a^1a^0 = a^0a^2 + a^2a^0 = a^0a^3 + a^3a^0 = a^1a^2 + a^2a^1 = a^1a^3 + a^3a^1 = a^2a^3 + a^3a^2 = 0
\]
Clearly these relations cannot be satisfied by scalars, so Dirac sought a set of four matrices which satisfy these relations.  These are now known as the Dirac matrices, and are given as follows:
\[
\gamma^0 = -ia^0 = 
\begin{pmatrix} 
1 & 0 & 0 & 0 \\
0 & 1 & 0 & 0 \\ 
0 & 0 & -1 & 0 \\
0 & 0 & 0 & -1 \end{pmatrix},
\gamma^1 = -ia^1 = 
\begin{pmatrix}
0 & 0 & 0 & 1 \\
0 & 0 & 1 & 0 \\
0 & -1 & 0 & 0 \\
-1 & 0 & 0 & 0 \end{pmatrix}
\]
\[
\gamma^2 = -ia^2 =
\begin{pmatrix}
0 & 0 & 0 & -i \\
0 & 0 & i & 0 \\
0 & i & 0 & 0 \\
-i & 0 & 0 & 0 \end{pmatrix},
\gamma^3 = a^3 =
\begin{pmatrix}
0 & 0 & 1 & 0 \\
0 & 0 & 0 & -1 \\
-1 & 0 & 0 & 0 \\
0 & 1 & 0 & 0 \end{pmatrix}
\]
These matrices are also known as the generators of the special unitary group of order 4, i.e. the group of $4 \times 4$ matrices with unit determinant.
Using these matrices, and switching to natural units ($\hbar = c = 1$) we can now obtain the Dirac equation:
\[
(\gamma^\mu \partial_\mu - im)\psi = 0
\]
\subsection{Feynman slash notation}
Richard Feynman developed the following convenient notation for terms involving Dirac matrices:
\[
\gamma^\mu q_\mu := \cancel{q}
\]
Using this notation, the Dirac equation is simply
\[
(\cancel{\partial} - im)\psi = 0
\]
\subsection{Relationship with Pauli matrices}
The Dirac matrices can be written more concisely as matrices of Pauli matrices, as follows:
\begin{align*}
\gamma_0 &= \begin{pmatrix} \sigma_0 & 0\\
                            0 & -\sigma_0
            \end{pmatrix}\\
\gamma_1 &= \begin{pmatrix} 0 & \sigma_1\\
                            -\sigma_1 & 0
            \end{pmatrix}\\
\gamma_2 &= \begin{pmatrix} 0 & \sigma_2\\
                            -\sigma_2 & 0
            \end{pmatrix}\\
\gamma_3 &= \begin{pmatrix} 0 & \sigma_3\\
                            -\sigma_3 & 0
            \end{pmatrix}
\end{align*}
%%%%%
%%%%%
\end{document}
