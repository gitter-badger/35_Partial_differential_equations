\documentclass[12pt]{article}
\usepackage{pmmeta}
\pmcanonicalname{ConnectionBetweenRiccatiEquationAndAiryFunctions}
\pmcreated{2013-03-22 18:09:07}
\pmmodified{2013-03-22 18:09:07}
\pmowner{perucho}{2192}
\pmmodifier{perucho}{2192}
\pmtitle{connection between Riccati equation and Airy functions}
\pmrecord{5}{40708}
\pmprivacy{1}
\pmauthor{perucho}{2192}
\pmtype{Derivation}
\pmcomment{trigger rebuild}
\pmclassification{msc}{35-00}
\pmclassification{msc}{34-00}

\endmetadata

% this is the default PlanetMath preamble.  as your knowledge
% of TeX increases, you will probably want to edit this, but
% it should be fine as is for beginners.

% almost certainly you want these
\usepackage{amssymb}
\usepackage{amsmath}
\usepackage{amsfonts}
\usepackage{amsthm}

% used for TeXing text within eps files
%\usepackage{psfrag}
% need this for including graphics (\includegraphics)
%\usepackage{graphicx}
% for neatly defining theorems and propositions
%\usepackage{amsthm}
% making logically defined graphics
%%%\usepackage{xypic}

% there are many more packages, add them here as you need them

% define commands here
\newtheorem{theorem}{Theorem}
\newtheorem{defn}{Definition}
\newtheorem{prop}{Proposition}
\newtheorem{lemma}{Lemma}
\newtheorem{cor}{Corollary}

\begin{document}
We report an interesting connection relating Riccati equation with Airy functions. Let us consider the nonlinear complex operator $\mathfrak{L}:z\in\mathbb{C}\mapsto\zeta$ with kernel given by
\begin{equation}
\frac{d\zeta}{dz}+\zeta^2+a(z)\zeta+b(z)=0,
\end{equation}
a nonlinear ODE of the first order so-called Riccati equation. In order to accomplish our purpose we particularize (1) by setting $a(z)\equiv 0$ and $b(z)=-z$. Thus (1) becomes
\begin{equation}
\frac{d\zeta}{dz}+\zeta^2=z.
\end{equation}
(2) can be reduced to a linear equation of the second order by the suitable change: $\zeta=w'(z)/w(z)$, whence
\begin{equation*} 
\zeta'=\frac{w''}{w}-\frac{w'^2}{w^2}, \qquad \zeta^2=\left(\frac{w'}{w}\right)^2,
\end{equation*}
which leads (2) to
\begin{equation}
w''-zw=0.
\end{equation}
Pairs of linearly independent solutions of (3) are the Airy functions.
  

%%%%%
%%%%%
\end{document}
