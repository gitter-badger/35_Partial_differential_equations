\documentclass[12pt]{article}
\usepackage{pmmeta}
\pmcanonicalname{Soliton}
\pmcreated{2013-03-22 17:47:50}
\pmmodified{2013-03-22 17:47:50}
\pmowner{invisiblerhino}{19637}
\pmmodifier{invisiblerhino}{19637}
\pmtitle{soliton}
\pmrecord{10}{40259}
\pmprivacy{1}
\pmauthor{invisiblerhino}{19637}
\pmtype{Definition}
\pmcomment{trigger rebuild}
\pmclassification{msc}{35Q51}
\pmclassification{msc}{37K40}
\pmsynonym{solitary wave}{Soliton}

\endmetadata

% this is the default PlanetMath preamble.  as your knowledge
% of TeX increases, you will probably want to edit this, but
% it should be fine as is for beginners.

% almost certainly you want these
\usepackage{amssymb}
\usepackage{amsmath}
\usepackage{amsfonts}

% used for TeXing text within eps files
%\usepackage{psfrag}
% need this for including graphics (\includegraphics)
%\usepackage{graphicx}
% for neatly defining theorems and propositions
%\usepackage{amsthm}
% making logically defined graphics
%%%\usepackage{xypic}

% there are many more packages, add them here as you need them

% define commands here

\begin{document}
A soliton is a non-linear object which moves through space without dispersion at constant speed. They occur naturally as solutions to the Korteweg - de Vries equation. They were first observed by John Scott Russell in the 19th century and then by Martin Kruskal and Norman Zabusky (who coined the term soliton) in a famous computer simulation in 1965. Insight into solitons can be obtained by noting that the Korteweg - de Vries equation satisfies D'Alembert's solution:
\[
u(x, t) = f(x-ct) + g(x+ct)
\]
We see at once that this satisfies two important criteria: it has a constant velocity $c$, and it can also be shown that the two functions $f$ and $g$ can collide without altering shape. Solitons also occur in non-linear optics and as solutions to field equations in quantum field theory.
%%%%%
%%%%%
\end{document}
