\documentclass[12pt]{article}
\usepackage{pmmeta}
\pmcanonicalname{TelegraphEquation}
\pmcreated{2013-03-22 18:03:15}
\pmmodified{2013-03-22 18:03:15}
\pmowner{pahio}{2872}
\pmmodifier{pahio}{2872}
\pmtitle{telegraph equation}
\pmrecord{13}{40580}
\pmprivacy{1}
\pmauthor{pahio}{2872}
\pmtype{Definition}
\pmcomment{trigger rebuild}
\pmclassification{msc}{35L15}
\pmclassification{msc}{35L20}
\pmsynonym{telegrapher's equation}{TelegraphEquation}
\pmrelated{HeavisideStepFunction}
\pmrelated{SecondOrderLinearDifferentialEquation}
\pmrelated{DelayTheorem}
\pmrelated{MellinsInverseFormula}
\pmrelated{TableOfLaplaceTransforms}

% this is the default PlanetMath preamble.  as your knowledge
% of TeX increases, you will probably want to edit this, but
% it should be fine as is for beginners.

% almost certainly you want these
\usepackage{amssymb}
\usepackage{amsmath}
\usepackage{amsfonts}

% used for TeXing text within eps files
%\usepackage{psfrag}
% need this for including graphics (\includegraphics)
%\usepackage{graphicx}
% for neatly defining theorems and propositions
 \usepackage{amsthm}
% making logically defined graphics
%%%\usepackage{xypic}

% there are many more packages, add them here as you need them

% define commands here

\theoremstyle{definition}
\newtheorem*{thmplain}{Theorem}

\begin{document}
Both the electric voltage and the \PMlinkescapetext{current in a double conductor} satisfy the {\em telegraph equation}
\begin{align}
f_{xx}''-af_{tt}''-bf_t'-cf = 0,
\end{align}
where $x$ is distance, $t$ is time and\, $a,\,b,\,c$\, are non-negative constants.\, The equation is a generalised form of the wave equation.\\

If the initial conditions are\, $f(x,\,0) = f_t'(x,\,0) = 0$\, and the boundary conditions \,$f(0,\,t) = g(t)$,\, $f(\infty,\,t) = 0$,\, then the Laplace transform of the solution function \,$f(x,\,t)$\, is
\begin{align}
F(x,\,s) = G(s)e^{-x\sqrt{as^2+bs+c}}.
\end{align}
In the special case\, $b^2-4ac = 0$,\, the solution is
\begin{align}
f(x,\,t) = e^{-\frac{bx}{2\sqrt{a}}}g(t-x\sqrt{a})H(t-x\sqrt{a}).
\end{align}

{\em Justification of} (2).\; Transforming the partial differential equation (1) ($x$ may be regarded as a parametre) gives
$$F_{xx}''(x,\,s)-a[s^2F(x,\,s)-sf(x,\,0)-f_t'(x,\,0)]-b[sF(x,\,s)-f(x,\,0)]-cF(x,\,s) = 0,$$
which due to the initial conditions simplifies to
$$F_{xx}''(x,\,s) = (\underbrace{as^2+bs+c}_{K^2})F(x,\,s).$$
The solution of this ordinary differential equation is
$$F(x,\,s) = C_1e^{Kx}+C_2e^{-Kx}.$$
Using the latter boundary condition, we see that
$$F(\infty,\,s) = \int_0^\infty e^{-st}f(\infty,\,t)\,dt \equiv 0,$$
whence\, $C_1 = 0$.\, Thus the former boundary condition implies
$$C_2 = F(0,\,s) = \mathcal{L}\{g(t)\} = G(s).$$
So we obtain the equation (2).\\

{\em Justification of} (3).\; When the \PMlinkname{discriminant}{QuadraticFormula} of the quadratic equation \,$as^2\!+\!bs\!+\!c = 0$\, vanishes, the \PMlinkname{roots}{Equation} coincide to\, $s = -\frac{b}{2a}$,\, and\, $as^2\!+\!bs\!+\!c = a(s+\frac{b}{2a})^2$.\, Therefore (2) reads
$$F(x,\,s) = G(s)a^{-x\sqrt{a}(s+\frac{b}{2a})} = e^{-\frac{bx}{2\sqrt{a}}}e^{-x\sqrt{a}s}G(s).$$
According to the delay theorem, we have
$$\mathcal{L}^{-1}\{e^{-ks}G(s)\} = g(t-k)H(t-k).$$
Thus we obtain for $\mathcal{L}^{-1}\{F(x,\,s)\}$ the expression of (3).
%%%%%
%%%%%
\end{document}
