\documentclass[12pt]{article}
\usepackage{pmmeta}
\pmcanonicalname{DAlembertAndDBernoulliSolutionsOfWaveEquation}
\pmcreated{2013-03-22 18:23:15}
\pmmodified{2013-03-22 18:23:15}
\pmowner{pahio}{2872}
\pmmodifier{pahio}{2872}
\pmtitle{d'Alembert and D. Bernoulli solutions of wave equation}
\pmrecord{11}{41031}
\pmprivacy{1}
\pmauthor{pahio}{2872}
\pmtype{Derivation}
\pmcomment{trigger rebuild}
\pmclassification{msc}{35L15}
\pmclassification{msc}{35L05}
\pmrelated{AdditionFormulasForSineAndCosine}
\pmrelated{SchrodingersWaveEquation}

\endmetadata

% this is the default PlanetMath preamble.  as your knowledge
% of TeX increases, you will probably want to edit this, but
% it should be fine as is for beginners.

% almost certainly you want these
\usepackage{amssymb}
\usepackage{amsmath}
\usepackage{amsfonts}

% used for TeXing text within eps files
%\usepackage{psfrag}
% need this for including graphics (\includegraphics)
%\usepackage{graphicx}
% for neatly defining theorems and propositions
 \usepackage{amsthm}
% making logically defined graphics
%%%\usepackage{xypic}

% there are many more packages, add them here as you need them

% define commands here

\theoremstyle{definition}
\newtheorem*{thmplain}{Theorem}

\begin{document}
Let's consider the \PMlinkname{d'Alembert's solution}{WaveEquation}
\begin{align}
u(x,\,t) \,:=\, \frac{1}{2}[f(x\!-\!ct)+f(x\!+\!ct)]+\frac{1}{2c}\int_{x-ct}^{x+ct}g(s)\,ds
\end{align}
of the wave equation in one dimension in the special case when the other initial condition is
\begin{align}
u'_t(x,\,0) \,:=\, g(x) \,\equiv\, 0.
\end{align}
We shall see that the solution is equivalent with the solution of D. Bernoulli.\\ \\

We \PMlinkescapetext{expand} the given function $f$ to the Fourier sine series on the interval \,$[0,\,p]$:
$$
f(y) \,=\, \sum_{n=1}^\infty A_n\sin\frac{n\pi y}{p} \quad \mbox{with}\;\; 
A_n = \frac{2}{p}\int_0^pf(x)\sin\frac{n\pi x}{p}\,dx \quad (n = 1,\,2,\,\ldots)
$$
Thus we may write
\begin{align*}
\begin{cases}
   f(x\!-\!ct) = \sum_{n=1}^\infty A_n\sin\!\left(\frac{n\pi x}{p}-\frac{n\pi ct}{p}\right)= 
\sum_{n=1}^\infty A_n\left(\sin\frac{n\pi x}{p}\cos\frac{n\pi ct}{p}-\cos\frac{n\pi x}{p}\sin\frac{n\pi ct}{p}\right),
\\ f(x\!+\!ct) = \sum_{n=1}^\infty A_n\sin\!\left(\frac{n\pi x}{p}+\frac{n\pi ct}{p}\right)= 
\sum_{n=1}^\infty A_n\left(\sin\frac{n\pi x}{p}\cos\frac{n\pi ct}{p}+\cos\frac{n\pi x}{p}\sin\frac{n\pi ct}{p}\right). 
\end{cases}
\end{align*}
Adding these equations and dividing by 2 yield
\begin{align}
u(x,\,t) = \frac{1}{2}[f(x\!-\!ct)+f(x\!+\!ct)] 
= \sum_{n=1}^\infty A_n\cos\frac{n\pi ct}{p}\sin\frac{n\pi x}{p},
\end{align}
which indeed is the \PMlinkname{solution of D. Bernoulli}{SolvingTheWaveEquationByDBernoulli} in the case\, $g(x) \equiv 0$.\\

\textbf{Note.}\, The solution (3) of the wave equation is especially \PMlinkescapetext{simple} in the special case where one has besides (2) the sine-formed initial condition
\begin{align}
u(x,\,0) \,:=\, f(x) \,\equiv\, \sin\frac{\pi x}{p}.
\end{align}
Then \,$A_n = 0$\, for every $n$ except 1, and one obtains
\begin{align}
u(x,\,t) \,= \cos\frac{\pi ct}{p}\sin\frac{\pi x}{p}\,.
\end{align}

\textbf{Remark.}\, 
In the case of quantum systems one has \PMlinkname{Schr\"odinger's wave equation}{SchrodingersWaveEquation}
whose solutions are different from the above. 
%%%%%
%%%%%
\end{document}
