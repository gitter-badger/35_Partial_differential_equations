\documentclass[12pt]{article}
\usepackage{pmmeta}
\pmcanonicalname{Hypoelliptic}
\pmcreated{2013-03-22 16:01:13}
\pmmodified{2013-03-22 16:01:13}
\pmowner{jirka}{4157}
\pmmodifier{jirka}{4157}
\pmtitle{hypoelliptic}
\pmrecord{7}{38059}
\pmprivacy{1}
\pmauthor{jirka}{4157}
\pmtype{Definition}
\pmcomment{trigger rebuild}
\pmclassification{msc}{35H10}
\pmdefines{analytically hypoelliptic}
\pmdefines{analytic hypoelliptic}

\endmetadata

% this is the default PlanetMath preamble.  as your knowledge
% of TeX increases, you will probably want to edit this, but
% it should be fine as is for beginners.

% almost certainly you want these
\usepackage{amssymb}
\usepackage{amsmath}
\usepackage{amsfonts}

% used for TeXing text within eps files
%\usepackage{psfrag}
% need this for including graphics (\includegraphics)
%\usepackage{graphicx}
% for neatly defining theorems and propositions
\usepackage{amsthm}
% making logically defined graphics
%%%\usepackage{xypic}

% there are many more packages, add them here as you need them

% define commands here
\theoremstyle{theorem}
\newtheorem*{thm}{Theorem}
\newtheorem*{lemma}{Lemma}
\newtheorem*{conj}{Conjecture}
\newtheorem*{cor}{Corollary}
\newtheorem*{example}{Example}
\newtheorem*{prop}{Proposition}
\theoremstyle{definition}
\newtheorem*{defn}{Definition}
\theoremstyle{remark}
\newtheorem*{rmk}{Remark}

\begin{document}
\begin{defn}
Let $P$ be a partial differential operator defined in an open subset $U \subset{\mathbb{R}}^n$.
If
for every \PMlinkname{distribution}{Distribution4} $u$ defined in an open subset $V \subset U$ such that
$Pu$ is $C^\infty$ (smooth), $u$ must also be $C^\infty$, then $P$ is
called {\em hypoelliptic}.
\end{defn}

Similarly, if the same assertion holds with $C^\infty$ replaced by
real analytic,
then $P$ is said to be {\em analytically hypoelliptic}.

Note that some authors use ``hypoelliptic'' to mean ``analytically hypoelliptic.''  Hence, if it is not clear from context, it is best to specify the regularity when using the term.  For example, $C^\infty$-hypoelliptic instead of just hypoelliptic.


\begin{thebibliography}{9}
\bibitem{netoartino}
J.\@ Barros-Neto, Ralph A.\@ Artino.
{\em \PMlinkescapetext{Hypoelliptic boundary value problems}},
Lecture Notes in Pure and Applied Mathematics, 53. Marcel Dekker, Inc., New York, 1980.
\PMlinkexternal{MR 81k:35031}{http://www.ams.org/mathscinet-getitem?mr=81k:35031}
\bibitem{heffernier}
Bernard Helffer, Francis Nier.
{\em \PMlinkescapetext{Hypoelliptic estimates and spectral theory for Fokker-Planck operators and Witten Laplacians}},
Lecture Notes in Mathematics, 1862. Springer-Verlag, Berlin, 2005.
\PMlinkexternal{MR 2006a:58039}{http://www.ams.org/mathscinet-getitem?mr=2006a:58039}
\bibitem{Shimakura:EPDE}
Norio Shimakura.
{\em \PMlinkescapetext{Elliptic Partial Differential Operators}},
Kinokuniya Company Ltd., Tokyo, Japan, 1978.
\end{thebibliography}

%%%%%
%%%%%
\end{document}
