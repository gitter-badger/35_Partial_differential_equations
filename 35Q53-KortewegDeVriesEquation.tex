\documentclass[12pt]{article}
\usepackage{pmmeta}
\pmcanonicalname{KortewegDeVriesEquation}
\pmcreated{2013-03-22 13:43:20}
\pmmodified{2013-03-22 13:43:20}
\pmowner{rspuzio}{6075}
\pmmodifier{rspuzio}{6075}
\pmtitle{Korteweg - de Vries equation}
\pmrecord{19}{34406}
\pmprivacy{1}
\pmauthor{rspuzio}{6075}
\pmtype{Definition}
\pmcomment{trigger rebuild}
\pmclassification{msc}{35Q53}
\pmsynonym{KdV equation}{KortewegDeVriesEquation}

% this is the default PlanetMath preamble.  as your knowledge
% of TeX increases, you will probably want to edit this, but
% it should be fine as is for beginners.

% almost certainly you want these
\usepackage{amssymb}
\usepackage{amsmath}
\usepackage{amsfonts}

% used for TeXing text within eps files
%\usepackage{psfrag}
% need this for including graphics (\includegraphics)
%\usepackage{graphicx}
% for neatly defining theorems and propositions
%\usepackage{amsthm}
% making logically defined graphics
%%%\usepackage{xypic}

% there are many more packages, add them here as you need them

% define commands here
\begin{document}
\PMlinkescapeword{model}

The \emph{Korteweg - de Vries equation} is a partial differential
equation defined as
\begin{equation}
u_t + 6u u_x + u_{xxx} = 0
\end{equation}
where $u = u(x,t)$ and the subscripts indicate derivatives.  This 
equation arises in hydrodynamics and was originally proposed to 
model waves in a canal. In addition to its practical applications, 
this equation is quite interesting as an object of mathematical 
study.  It exhibits interesting soliton solutions, has a large
algebra of conserved quantities, and can be solved using methods 
of inverse scattering.

\section{Travelling Wave Solution}

It is easy to exhibit a solution which describes a traveling wave.
To find this solution, one substitutes the ansatz $u(x,t)=f(x-ct)$
\footnote{The most general form of a solution to the two-dimensional
wave equation is $u(x,t) = f(x-ct) + f(x+ct)$ which describes waves 
propagating in both directions with velocity $c$.  Here, we will
look for a solution which only describes a wave propagatng in one
direction.} 
into the equation to obtain the following equation for $f$:
 \[-cf' + 6f f' + f''' = 0\]
This equation can be written as 
 \[(-cf + 3 f^2 + f'')' = 0\]
and, hence,
 \[-cf + 3 f^2 + f'' = k\]
for some constant $k$.  Multiplying by $f'$, we can repeat the same
trick:
 \[-cff' + 3 f^2 f' + f'f'' - kf' =
(-cf^2/2 + f^3 + (f')^2 - kf)' = 0\]
hence
\[-cf^2/2 + f^3 + (f')^2 - kf = h\]
or
\[f' = \sqrt{ kf + cf^2/2 - f^3}\]
This can be solved implicitly by an integral:
\[y = \int^{f(y)} \sqrt{ kf + cf^2/2 - f^3} \,dy + C\]
Since this is an elliptic integral, the result is an elliptic function.

The solution obtained above is know as a \emph{solitary wave}, or \emph{soliton}.
This term ``solitary wave'' refers to the fact that this solution describes
a single wave pulse traveling with velocity $c$.  Note that the amplitude of
the pulse is determined by its velocity, unlike in the case of linear wave 
equations where the velocity of propagation does not depend upon the amplitude.
There are also solutions which describe more than one solitary wave.  In 
particular, there are solutions in which two of these waves collide and then
re-emerge from the collision.

\section{Conserved Currents}

It is possible to exhibit integrals whose value is conserved under the
evolution.  To construct these quantities, we begin with vector fields.
Suppose that $u$ is a thrice differentiable function of $x$ and $t$ and
consider the following vector fields:
\begin{align*}
 v^{(1)} &= (u, 3 u^2 + u_{xx})  \\
 v^{(2)} &= (u^2, 4 u^3 - u_x^2 + 2 u u_{xx})
\end{align*}
Computing their divergences and rearranging the result,
\begin{align*}
 {\partial v^{(1)}_t \over \partial t} +
 {\partial v^{(1)}_t \over \partial x} &=
 u_{t} + 6 u u_x + u_{xxx} \\
 {\partial v^{(2)}_t \over \partial t} +
 {\partial v^{(2)}_t \over \partial x} &=
 2 u (u_t + 6 u u_x + u_{xxx}) .
\end{align*}
If $u$ happens to satisfy our differential equation, then the
divergence of these vector fields will go zero.

To obtain the conserved quantities, we will integrate them over a rectangle in
the $t,x$ plane with sides parallel to the $x$ and $t$ axes use Green's theorem .
\begin{align*}
 \int_{t_1}^{t_2} dt \, \int_{x_1}^{x_2} dx \,
 &(u_{t}(t,x) + 6 u(t,x) u_x(t,x) + u_{xxx}(t,x)) \\ = 
 &\int_{x_1}^{x_2} dx \, (3 u^2(x,t_2) + u_{xx}(x,t_2)) +
 \int_{t_2}^{t_1} dt \, u(t,x_2) + \\
 &\int_{x_1}^{x_2} dx \, (3 u^2(x,t_1) + u_{xx}(x,t_1)) +
 \int_{t_1}^{t_2} dt \, u(t,x_1) \\ =
 &3 \int_{x_1}^{x_2} dx \, u^2(x,t_2) -
 3 \int_{x_1}^{x_2} dx \, u^2(x,t_1) \\ +
 &\int_{t_1}^{t_2} dt \, u(t,x_1) -
 \int_{t_1}^{t_2} dt \, u(t,x_2) +
 u_x(x_2,t_2) - u_x(x_1,t_2) - u_x(x_2,t_1) + u_x(x_1,t_1) \\
 \int_{t_1}^{t_2} dt \, \int_{x_1}^{x_2} dx \,
 &2 u (u_{t}(t,x) + 6 u(t,x) u_x(t,x) + u_{xxx}(t,x)) \\ = 
 &\int_{x_1}^{x_2} dx \, (4 u^3(x,t_2) - u_x^2(x,t_2) + 2 u(x,t_2) u_{xx}(x,t_2)) +
 \int_{t_2}^{t_1} dt \, u^2(t,x_2) + \\
 &\int_{x_2}^{x_2} dx \, (4 u^3(x,t_1) - u_x^2(x,t_1) + 2 u(x,t_1) u_{xx}(x,t_1)) +
 \int_{t_1}^{t_2} dt \, u^2(t,x_1) \\ =
 &\int_{x_1}^{x_2} dx \, (4 u^3(x,t_2) - 3 u_x^2(x,t_2)) -
 \int_{x_1}^{x_2} dx \, (4 u^3(x,t_1) - 3 u_x^2(x,t_1)) \\ +
 &\int_{t_1}^{t_2} dt \, u^2(t,x_1) -
 \int_{t_1}^{t_2} dt \, u^2(t,x_2) \\ +
 &u(x_2,t_2) u_x(x_2,t_2) - u (x_1,t_2) u_x(x_1,t_2) - 
 u(x_2,t_1) u_x(x_2,t_1) + u(x_1,t_1) u_x(x_1,t_1)
\end{align*}
We consider now several popular boundary conditions for our equation:
\begin{itemize}
\item{\bf Periodic}  Suppose that $u$ is periodic in $x$ with period $p$ and
satisfies the Kortwieg - de Vries equation.  Then we integrals over $t$
cancel against each other as do the endpoint terms, leaving us with
\[
 \int_{x_1}^{x_2} dx \, u^2(x,t_1) = 
 \int_{x_1}^{x_2} dx \, u^2(x,t_2).
\]
\item{\bf Whole line}  Suppose that $u$ satisfies the Kortwieg - de Vries equation
and that $u$ and its derivatives tend towards zero as $x \to \pm \infty$.  Then,
taking the limit as $x_1 \to -\infty$ and $x_2 \to + \infty$, we conclude that
\[
 \int_{-\infty}^{+\infty} dx \, u^2(x,t_1) = 
 \int_{-\infty}^{+\infty} dx \, u^2(x,t_2) .
\]
\item{Finite interval}  We may also consider our differential equation on
a finite interval and impose suitable boundary conditions at the endpoints.
In this case, the integrals with respect to $t$ do not cancel or automatically 
vanish as a consequence of the boundary conditions, so they will not, in general, 
give conserved quantities.  Nevertheless, it is at least still possible to compute 
their value solely from the boundary data without having to know how the solution 
behaves in the interior.
\end{itemize}

Starting with the two vector fields given above, it is possible to generate
more such vector fields and more conserved quantities.  Since the Lie bracket
of two vector fields with zero divergence also has zero divergence, we may

%%%%%
%%%%%
\end{document}
