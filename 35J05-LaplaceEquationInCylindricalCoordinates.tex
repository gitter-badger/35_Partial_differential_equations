\documentclass[12pt]{article}
\usepackage{pmmeta}
\pmcanonicalname{LaplaceEquationInCylindricalCoordinates}
\pmcreated{2013-03-22 16:24:24}
\pmmodified{2013-03-22 16:24:24}
\pmowner{bloftin}{6104}
\pmmodifier{bloftin}{6104}
\pmtitle{Laplace equation in cylindrical coordinates}
\pmrecord{13}{38555}
\pmprivacy{1}
\pmauthor{bloftin}{6104}
\pmtype{Derivation}
\pmcomment{trigger rebuild}
\pmclassification{msc}{35J05}
\pmrelated{BesselFunction}
\pmrelated{BesselsEquation}

% this is the default PlanetMath preamble.  as your knowledge
% of TeX increases, you will probably want to edit this, but
% it should be fine as is for beginners.

% almost certainly you want these
\usepackage{amssymb}
\usepackage{amsmath}
\usepackage{amsfonts}

% used for TeXing text within eps files
%\usepackage{psfrag}
% need this for including graphics (\includegraphics)
%\usepackage{graphicx}
% for neatly defining theorems and propositions
%\usepackage{amsthm}
% making logically defined graphics
%%%\usepackage{xypic}

% there are many more packages, add them here as you need them

% define commands here

\begin{document}
\section{Laplace Equation in Cylindrical Coordinates}

Solutions to the Laplace equation in cylindrical coordinates have wide applicability from fluid mechanics to electrostatics.
Applying the method of separation of variables to Laplace's partial differential equation and then enumerating the various forms
of solutions will lay down a foundation for solving problems in this coordinate system.  Finally, the use of Bessel functions
in the solution reminds us why they are synonymous with the cylindrical domain.

\subsection{Separation of Variables}
Beginning with the Laplacian in cylindrical coordinates, apply the operator to a potential function and set it equal to zero to get the Laplace equation

\begin{equation}
\nabla^{2} \Phi = \frac{1}{r} \frac{\partial}{\partial r}\left(r \frac{\partial\Phi}{\partial r}\right) + \frac{1}{r^2} \frac{\partial^2\Phi}{\partial \theta^2} + \frac{\partial^2 \Phi}{\partial z^2} = 0.
\end{equation}

First expand out the terms

\begin{equation}
\nabla^{2} \Phi = \frac{1}{r} \frac{\partial \Phi}{\partial r} + \frac{\partial^2\Phi}{\partial r^2} + \frac{1}{r^2} \frac{\partial^2\Phi}{\partial \theta^2} + \frac{\partial^2 \Phi}{\partial z^2} = 0.
\end{equation}

Then apply the method of separation of variables by assuming the solution is in the form

$$ \Phi \left ( r,\theta,z \right) = R(r)P(\theta)Z(z).$$

Plug this into (2) and note how we can bring out the functions that are not affected by the derivatives

$$ \frac{P Z}{r} \frac{d R}{d r} + P Z \frac{d^2 R}{d r^2} + \frac{R Z}{r^2} \frac{d^2 P}{d \theta^2}  + R P \frac{d^2 Z}{d z^2} = 0.$$

Divide by $R(r) P(\theta) Z(z)$ and use short hand notation to get

$$\frac{1}{Rr} \frac{d R}{d r} + \frac{1}{R} \frac{d^2R}{d r^2} + \frac{1}{Pr^2} \frac{d^2P}{d \theta^2} + \frac{1}{Z} \frac{d^2 Z}{dz^2} = 0.$$

``Separating'' the z term to the other side gives

$$\frac{1}{Rr} \frac{d R}{d r} + \frac{1}{R} \frac{d^2R}{d r^2} + \frac{1}{Pr^2} \frac{d^2P}{d \theta^2} = - \frac{1}{Z} \frac{d^2 Z}{dz^2}.$$

This equation can only be satisfied for all values if both sides are equal to a constant, $\lambda$, such that

\begin{equation}
-\frac{1}{Z} \frac{d^2 Z}{dz^2} = \lambda
\end{equation}

\begin{equation}
\frac{1}{Rr} \frac{d R}{d r} + \frac{1}{R} \frac{d^2R}{d r^2} + \frac{1}{Pr^2} \frac{d^2P}{d\theta^2} = \lambda.
\end{equation}

Before we can focus on solutions, we need to further separate (4), so multiply (4) by $r^2$

$$\frac{r}{R} \frac{d R}{d r} + \frac{r^2}{R} \frac{d^2R}{d r^2} + \frac{1}{P} \frac{d^2P}{d\theta^2} = \lambda r^2.$$

Separate the terms

$$\frac{r}{R} \frac{d R}{d r} + \frac{r^2}{R} \frac{d^2R}{d r^2} - \lambda r^2 = - \frac{1}{P} \frac{d^2P}{d \theta^2}.$$

As before, set both sides to a constant, $\kappa$

\begin{equation}
- \frac{1}{P} \frac{d^2P}{d \theta^2}= \kappa
\end{equation}

\begin{equation}
\frac{r}{R} \frac{dR}{d r} + \frac{r^2}{R} \frac{d^2R}{d r^2} - \lambda r^2  = \kappa.
\end{equation}

Now there are three differential equations and we know the form of these solutions.  The differential equations of
(3) and (5) are ordinary differential equations, while (6) is a little more complicated and we must turn to Bessel functions.

\subsection{Axial Solutions ($z$)}

Following the guidelines setup in [Etgen] for linear homogeneous differential equations, the first step in solving

$$\frac{d^2 Z}{dz^2} + Z \lambda = 0$$

is to find the roots of the characteristic polynomial

$$C(r) = r^2 + \lambda = 0$$

$$r =  \pm \sqrt{ -\lambda}.$$

Although, one can go forward using the square root, here we will introduce another constant, $\gamma$ to imply
the following cases.  So if we want real roots, then we want to ensure a negative constant

$$ \lambda = -\gamma^2 $$

and if we want complex roots, then we want to ensure a positive constant

$$ \lambda = \gamma^2 ,$$

{\bf Case 1}: $\lambda \le 0$ and real roots $(\lambda = -\gamma^2)$.

For every real root, there will be an exponential in the general solution.  The real roots are

$$ r_1 = \gamma $$
$$ r_2 = -\gamma $$
$$ r_3 = 0 .$$

Therefore, the solutions for these roots are

$$ h_1(z) = C_1e^{\gamma z} $$
$$ h_2(z) = C_2e^{-\gamma z} $$
$$ h_3(z) = C_3z e^0  = C_3z.$$

Combining these using the principle of superposition, gives the general solution,

\begin{equation}
Z_{\gamma}(z) =  C_1e^{\gamma z} + C_2e^{-\gamma z} + C_3z + C_4.
\end{equation}

{\bf Case 2}: $\lambda > 0$ and complex roots $(\lambda = \gamma^2)$.

The roots are
$$ r_4 = i \gamma $$
$$ r_5 = -i \gamma $$

and the corresponding solutions

$$ h_4(z) = C_5e^0 \cos (\gamma z) =  C_5\cos (\gamma z) $$
$$ h_5(z) = C_6e^0 \sin (\gamma z)=  C_6\sin (\gamma z) .$$

Combining these into a general solution yields

$$Z_{\gamma}(z) =  C_5\cos (\gamma z) + C_6\sin (\gamma z) + C_7.$$

\subsection{Azimuthal Solutions ($\theta$)}

Azimuthal solutions for

$$ \frac{d^2P}{d \theta^2} + \kappa P  = 0$$

are in the most general sense obtained similarly  to the axial solutions with the characteristic polynomial

$$C(r) = r^2 + \kappa = 0$$

$$r = \pm \sqrt{- \kappa}.$$

Using another constant, $\nu$ to ensure positive or negative constants, we get two cases.

{\bf Case 1}: $\kappa \le 0$ and real roots $(\kappa = -\nu^2)$.

The solutions for these roots are then

$$ h_1(z) = C_1e^{\nu \theta} $$
$$ h_2(z) = C_2e^{-\nu \theta} $$
$$ h_3(z) = C_3\theta e^0  = C_3\theta.$$

Combining these for the general solution,

\begin{equation}
P_{\nu}(\theta) =  C_1e^{\nu \theta} + C_2e^{-\nu \theta} + C_3\theta + C_4.
\end{equation}

{\bf Case 2}: $\kappa > 0$ and complex roots $(\kappa = \nu^2)$.

The roots are

$$ r_4 = i \nu $$
$$ r_5 = -i \nu $$

and the corresponding solutions

$$ h_4(\theta) = C_5e^0 \cos (\nu \theta) =  C_5\cos (\nu \theta) $$
$$ h_5(\theta) = C_6e^0 \sin (\nu \theta)=  C_6\sin (\nu \theta) .$$

Combining these into a general solution

$$P_{\nu}(\theta) =  C_5\cos (\nu \theta) + C_6\sin (\nu \theta) + C_7.$$

For the first glimpse at simplification, we will note a restriction on $\kappa$
that is used when it is required that the solution be periodic to ensure $P$ is single valued

$$P(\theta) = P(\theta + 2 n \pi).$$

Then we are left with either the periodic solutions that occur with complex roots or the zero case.  So not only

$$(\kappa = \nu^2)$$

but also $\nu$ must be an integer, i.e.

\begin{equation}
P_{\nu}(\theta) =  C_5 \cos (\nu \theta) + C_6 \sin (\nu \theta) + C_7 \,\,\,\,\,\,\,\,\,\,\, \nu = 0, 1, 2, ...
\end{equation}

Note, that $\nu = 0$, is still a solution, but to be periodic we can only have a constant

$$P(\theta) = C_4.$$

\subsection{Radial Solutions ($r$)}

The radial solutions are the more difficult ones to understand for this problem and are solved using a power series.
The two types of solutions generated based on the choices of constants from the $\theta$ and $z$ solutions (excluding non-periodic solutions for $P$)
leads to the Bessel functions and the modified Bessel functions.  The first step for both these cases is to transform (6)
into the Bessel differential equation.

{\bf Case 1}: $\lambda < 0$ $(\lambda = -\gamma^2)$, $\kappa > 0$ $(\kappa = \nu^2)$.

Substitute $\gamma$ and $\nu$ into the radial equation (6) to get

\begin{equation}
\frac{r}{R} \frac{dR}{d r} + \frac{r^2}{R} \frac{d^2R}{d r^2} + \gamma^2 r^2  - \nu^2 = 0.
\end{equation}

Next, use the substitution

$$x = \gamma r$$
$$r = \frac{x}{\gamma}.$$

Therefore, the derivatives are

$$dx = \gamma dr$$
$$dr = \frac{dx}{\gamma}$$

and make a special note that

$$\frac{d^2}{dx^2} = \frac{d}{dx} \frac{d}{dx}$$

so

$$dx^2 = dx*dx = \gamma^2 dr^2$$
$$dr^2 = \frac{dx^2}{\gamma^2}.$$

Substituting these relationships into (10) gives us

$$\frac{x \gamma}{\gamma R} \frac{dR}{dx} + \frac{x^2 \gamma^2}{\gamma^2 R} \frac{d^2R}{dx^2} + x^2  - \nu^2 = 0.$$

Finally, multiply by $R/x^2$ to get the \emph{Bessel differential equation}

\begin{equation}
\frac{d^2R}{dx^2} + \frac{1}{x} \frac{dR}{dx} +  \left(1  - \frac{\nu^2}{x^2} \right )R = 0.
\end{equation}

Delving into all the nuances of solving Bessel's differential equation is beyond the scope of this article, however, the curious
are directed to Watson's in depth treatise [Watson].  Here, we will just present the results as we did for the previous
differential equations.  The general solution is a linear combination of the Bessel function of the first kind $J_{\nu}(r)$
and the Bessel function of the second kind $Y_{\nu}(r)$.  Remebering  that $\nu$ is a positive integer or zero.

\begin{equation}
R_{\nu}(r) = C_1 J_{\nu}(\gamma  r) + C_2 Y_{\nu}(\gamma  r) + C_3
\end{equation}

Bessel function of the first kind:

$$J_{\nu}(x) = \sum_{m=0}^{\infty} \frac{ (-1)^m ( \frac{-1}{2}x)^{\nu + 2m} }{ m! (m + \nu)! }.$$

Bessel function of the second kind (using Hankel's formula):

$$Y_{\nu}(x) = 2J_{\nu}(x) \left (\eta + ln \left( \frac{x}{2}\right ) \right ) - \left( \frac{x}{2} \right)^{-\nu} \sum_{m=0}^{\nu-1} \frac{(\nu - m - 1)!}{m!} \left ( \frac{x}{2} \right)^{2m}$$
$$ \,\,\,\,\, -\sum_{m=0}^{\infty} \frac{ (-1)^m (\frac{x}{2})^{\nu + 2m} }{m! (\nu + m)!} \left \{ \frac{1}{1} + \frac{1}{2} + ... + \frac{1}{m}+ \frac{1}{1} + \frac{1}{2} + ...  + \frac{1}{\nu + m} \right \}.$$

For the unfortunate person who has to evaluate this function, note that when $m = 0$, the singularity is taken care of by replacing the series in brackets by

$$ \left \{ \frac{1}{1} + \frac{1}{2} + ... + \frac{1}{\nu} \right \}.$$

Some solace can be found since most physical problems need to be analytic at $x = 0$ and therefore $Y_{\nu}(x)$ breaks down at $ln(0)$.
This leads to the choice of constant $C_2$ to be zero.


{\bf Case 2}: $\lambda > 0$ $(\lambda = \gamma^2)$, $\kappa > 0$ $(\kappa = \nu^2)$.

Using the previous method of substitution, we just get the change of sign

\begin{equation}
\frac{d^2R}{dx^2} + \frac{1}{x} \frac{dR}{dx} -  \left(1 + \frac{\nu^2}{x^2} \right)R = 0.
\end{equation}

This leads to the modified Bessel functions as a solution, which are also known as the pure imaginary Bessel functions.  The
general solution is denoted

\begin{equation}
R_{\nu}(r) = C_1 I_{\nu}(\gamma  r) + C_2 K_{\nu}(\gamma  r) + C_3
\end{equation}

where $I_{\nu}$ is the modified Bessel function of the first kind and $K_{\nu}$ is the modified Bessel function of the
second kind

$$I_{\nu}(\gamma r) = i^{-\nu} J_{\nu}(i \gamma r)$$
$$K_{\nu}(\gamma r) = \frac{\pi}{2} i^{\nu+1} \left ( J_{\nu}(i \gamma r) + iY_{\nu}(i \gamma r) \right ) .$$

\subsection{Combined Solution}

Keeping track of all the different cases and choosing the right terms for boundary conditions is a daunting task when one attempts
to solve Laplace's equation.  The short hand notation used in [Kusse] and [Arfken] will be presented here
to help organize the choices as a reference.  It is important to remember that these solutions are only for
the single valued azimuth cases $(\kappa = \nu^2)$.

Once the separate solutions are obtained, the rest is simple since our solution is separable

$$ \Phi \left ( r,\theta,z \right) = R(r)P(\theta)Z(z).$$

So we just combine the individual solutions to get the general solutions to the Laplace equation in cylindrical coordinates.

{\bf Case 1}: $\lambda < 0$ $(\lambda = -\gamma^2)$, $\kappa > 0$ $(\kappa = \nu^2)$.

\begin{equation}
\Phi \left ( r,\theta,z \right) =  \sum_{\nu} \sum_{\gamma} \left \{ \begin{array}{c}
        e^{\gamma z} \\
        e^{-\gamma z}
           \end{array} \right .
     \left \{ \begin{array}{c}
        \cos (\nu \theta) \\
        \sin (\nu \theta)
           \end{array} \right .
        \left \{ \begin{array}{c}
        J_{\nu}(\gamma  r) \\
        Y_{\nu}(\gamma  r)
           \end{array} \right .
\end{equation}

{\bf Case 2}: $\lambda > 0$ $(\lambda = \gamma^2)$, $\kappa > 0$ $(\kappa = \nu^2)$.

\begin{equation}
\Phi \left ( r,\theta,z \right) =  \sum_{\nu} \sum_{\gamma} \left \{ \begin{array}{c}
        \cos (\gamma z) \\
        \sin (\gamma z)
           \end{array} \right .
     \left \{ \begin{array}{c}
        \cos (\nu \theta) \\
        \sin (\nu \theta)
           \end{array} \right .
        \left \{ \begin{array}{c}
        I_{\nu}(\gamma  r) \\
        K_{\nu}(\gamma  r)
           \end{array} \right .
\end{equation}

Interpreting the short hand notation is as simple as expanding terms and not forgetting the linear solutions, i.e. $(\gamma = 0)$ .
As an example, case 1, expanded out while ignoring the linear terms would give

\begin{equation}
\Phi \left ( r,\theta,z \right) =  \sum_{\nu} \sum_{\gamma}  \left \{ \begin{array}{c}
   \,  A_{\nu \gamma} e^{\gamma z} \cos (\nu \theta) J_{\nu}(\gamma  r) \\
   + B_{\nu \gamma} e^{\gamma z} \cos (\nu \theta) Y_{\nu}(\gamma  r) \\
   + C_{\nu \gamma} e^{\gamma z} \sin (\nu \theta) J_{\nu}(\gamma  r) \\
   + D_{\nu \gamma} e^{\gamma z} \sin (\nu \theta) Y_{\nu}(\gamma  r) \\
   + E_{\nu \gamma} e^{-\gamma z} \cos (\nu \theta) J_{\nu}(\gamma  r) \\
   + F_{\nu \gamma} e^{-\gamma z} \cos (\nu \theta) Y_{\nu}(\gamma  r) \\
   + G_{\nu \gamma} e^{-\gamma z} \sin (\nu \theta) J_{\nu}(\gamma  r) \\
   + H_{\nu \gamma} e^{-\gamma z} \sin (\nu \theta) Y_{\nu}(\gamma  r) \\

     \end{array} \right \} .
\end{equation}



\begin{thebibliography}{9}
\bibitem{Arfken} Arfken, George, Weber, Hans, {\em Mathematical Physics}. Academic Press, San Diego, 2001.
\bibitem{Etgen} Etgen, G., {\em Calculus}. John Wiley \& Sons, New York, 1999.
\bibitem{Guterman} Guterman, M., Nitecki, Z., {\em Differential Equations, 3rd Edition}. Saunders College Publishing, Fort Worth, 1992.
\bibitem{Jackson} Jackson, J.D., {\em Classical Electrodynamics, 2nd Edition}. John Wiley \& Sons, New York, 1975.
\bibitem{Kusse} Kusse, Bruce, Westwig, Erik, {\em Mathematical Physics}. John Wiley \& Sons, New York, 1998.
\bibitem{Lebedev} Lebedev, N., {\em Special Functions \& Their Applications}. Dover Publications, New York, 1995.
\bibitem{Watson} Watson, G.N., {\em A Treatise on the Theory of Bessel Functions}. Cambridge University Press, New York, 1995.
\end{thebibliography}

%%%%%
%%%%%
\end{document}
