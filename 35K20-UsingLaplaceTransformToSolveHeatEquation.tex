\documentclass[12pt]{article}
\usepackage{pmmeta}
\pmcanonicalname{UsingLaplaceTransformToSolveHeatEquation}
\pmcreated{2015-05-30 6:55:05}
\pmmodified{2015-05-30 6:55:05}
\pmowner{pahio}{2872}
\pmmodifier{pahio}{2872}
\pmtitle{using Laplace transform to solve heat equation}
\pmrecord{11}{41836}
\pmprivacy{1}
\pmauthor{pahio}{2872}
\pmtype{Example}
\pmcomment{trigger rebuild}
\pmclassification{msc}{35K20}
\pmclassification{msc}{35Q99}
\pmclassification{msc}{35K05}
\pmsynonym{using Laplace transform to solve partial differential equation}{UsingLaplaceTransformToSolveHeatEquation}
%\pmkeywords{PDE}
\pmrelated{LaplaceTransform}

\endmetadata

% this is the default PlanetMath preamble.  as your knowledge
% of TeX increases, you will probably want to edit this, but
% it should be fine as is for beginners.

% almost certainly you want these
\usepackage{amssymb}
\usepackage{amsmath}
\usepackage{amsfonts}

% used for TeXing text within eps files
%\usepackage{psfrag}
% need this for including graphics (\includegraphics)
%\usepackage{graphicx}
% for neatly defining theorems and propositions
 \usepackage{amsthm}
% making logically defined graphics
%%%\usepackage{xypic}

% there are many more packages, add them here as you need them

% define commands here

\theoremstyle{definition}
\newtheorem*{thmplain}{Theorem}

\begin{document}
Along the whole positive $x$-axis, we have an \PMlinkescapetext{ideal} heat-conducting rod, the surface of which is \PMlinkescapetext{isolated}.\, The initial temperature of the rod is 0 \PMlinkescapetext{degrees}.\, Determine the temperature function 
\,$u(x,\,t)$\, when at the time \,$t = 0$
 
(a) the head \,$x = 0$\, of the rod is set permanently to the constant temperature;

(b) through the head \,$x = 0$\, one directs a constant heat flux.\\


The heat equation in one dimension reads
\begin{align}
u_{xx}''(x,\,t) \;=\; \frac{1}{c^2}\,u_t'(x,\,t).
\end{align}
In this we have
\begin{align*}
\mbox{(a) }
\begin{cases}
\mbox{boundary conditions}\;\, u(\infty,\,t) = 0, \qquad u(0,\,t) = u_0, \\
\mbox{initial conditions} \qquad u(x,\,0) = 0, \qquad \underbrace{u_t'(x,\,0) = 0}_{\mbox{for\;} x\,>\,0} 
\end{cases}
\end{align*}
and
\begin{align*}
\mbox{(b) }
\begin{cases}
\mbox{boundary conditions}\;\, u(\infty,\,t) = 0, \quad u_x'(0,\,t) = -k, \\
\mbox{initial conditions} \qquad u(x,\,0) = 0, \quad \underbrace{u_t'(x,\,0) = 0}_{\mbox{for\;} x\,>\,0}. 
\end{cases}
\end{align*}

For solving (1), we first form its Laplace transform (see the table of Laplace transforms)
$$U_{xx}''(x,\,s) \;=\; \frac{1}{c^2}[s\,U(x,\,s)-u(x,\,0)],$$
which is a \PMlinkescapetext{simple} ordinary linear differential equation
$$U_{xx}''(x,\,s) \;=\; \left(\frac{\sqrt{s}}{c}\right)^2U(x,\,s)$$
of \PMlinkname{order}{ODE} two.\, Here, $s$ is only a parametre, and the general solution of the equation is 
$$U(x,\,s) \;=\; C_1e^{\frac{\sqrt{s}}{c}x}+C_2e^{-\frac{\sqrt{s}}{c}x}$$
(see \PMlinkname{this entry}{SecondOrderLinearODEWithConstantCoefficients}).\, Since 
$$U(\infty,\,s) \;=\; \int_0^\infty\!e^{-st}u(\infty,\,t)\,dt \;=\;\int_0^\infty\!0\,dt \;\equiv\; 0,$$
we must have\, $C_1 = 0$.\, Thus the Laplace transform of the solution of (1) is in both cases (a) and (b)
\begin{align}
U(x,\,s) \;=\; C_2e^{-\frac{\sqrt{s}}{c}x}.
\end{align}


For (a), the second boundary condition implies \, $\displaystyle U(0,\,s) = \frac{u_0}{s}$.\, But by (2) we must have\, $U(0,\,s) = C_2\!\cdot\!1$, whence we infer that\, $\displaystyle C_2 = \frac{u_0}{s}$.\, Accordingly, 
$$U(x,\,s) \;=\; u_0\cdot\frac{1}{s}e^{-\frac{x}{c}\sqrt{s}},$$
which corresponds to the solution function
$$u(x,\,t) \;:=\; u_0\mbox{ erfc}\frac{x}{2c\sqrt{t}}$$
of the heat equation (1).\\

For (b), the second boundary condition says that\, $\displaystyle U_x'(0,\,s) = -\frac{k}{s}$,\, and since (2) implies that\, 
$U_x'(x,\,s) =-\frac{\sqrt{s}}{c}C_2e^{-\frac{\sqrt{s}}{c}x}$,\, we can infer that now\, 
$$C_2 \;=\; \frac{ck}{s\sqrt{s}}.$$
Thus
$$U(x,\,s) \;=\; \frac{ck}{s\sqrt{s}}e^{-\frac{x}{c}\sqrt{s}},$$
which corresponds to
$$u(x,\,t) \;:=\; k\left[2c\sqrt{\frac{t}{\pi}}e^{-\frac{x^2}{4c^2t}}-x\mbox{ erfc}\frac{x}{2c\sqrt{t}}\right]\!.$$\\





%%%%%
%%%%%
\end{document}
