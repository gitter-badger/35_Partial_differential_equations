\documentclass[12pt]{article}
\usepackage{pmmeta}
\pmcanonicalname{VibratingStringWithVariableDensity}
\pmcreated{2013-03-22 17:26:42}
\pmmodified{2013-03-22 17:26:42}
\pmowner{perucho}{2192}
\pmmodifier{perucho}{2192}
\pmtitle{vibrating string with variable density}
\pmrecord{9}{39826}
\pmprivacy{1}
\pmauthor{perucho}{2192}
\pmtype{Example}
\pmcomment{trigger rebuild}
\pmclassification{msc}{35L05}

% this is the default PlanetMath preamble.  as your knowledge
% of TeX increases, you will probably want to edit this, but
% it should be fine as is for beginners.

% almost certainly you want these
\usepackage{amssymb}
\usepackage{amsmath}
\usepackage{amsfonts}

% used for TeXing text within eps files
%\usepackage{psfrag}
% need this for including graphics (\includegraphics)
%\usepackage{graphicx}
% for neatly defining theorems and propositions
%\usepackage{amsthm}
% making logically defined graphics
%%%\usepackage{xypic}

% there are many more packages, add them here as you need them

% define commands here
\newtheorem{theorem}{Theorem}
\newtheorem{defn}{Definition}
\newtheorem{prop}{Proposition}
\newtheorem{lemma}{Lemma}
\newtheorem{cor}{Corollary}

\begin{document}
The unidimensional wave's problem may be stated as
\begin{equation*}
\frac{\partial^2u}{\partial t^2}-c^2(x)\frac{\partial^2u}{\partial x^2}=f(x,t), \qquad x\in(0,l),\quad t>0,
\end{equation*}
with initial conditions
\begin{equation*}
\begin{cases}
\,\, u(x,0)=f(x), \\
\,\, \frac{\partial u}{\partial t}(x,0)=g(x),
\end{cases}
\end{equation*}
and boundary conditions
\begin{equation*}
\begin{cases}
\,\, u(0,t)=0, \\
\,\, u(l,t)=0,
\end{cases}
\end{equation*}
which may be specialized to a string's motion if we physically interpret $c^2(x)=T_0/\rho(x)$ as the ratio between the string's initial tension and its linear density. We will discuss the free string's vibrations (i.e. $f(x,t)\equiv 0$) given by the string's problem
\begin{equation}
\frac{\partial^2 u}{\partial t^2}-(1+x)^2\frac{\partial^2 u}{\partial x^2}=0, \qquad x\in (0,1), \quad t>0,
\end{equation}
initial conditions
\begin{equation*}
\begin{cases}
\,\, u(x,0)=f(x), & \textrm{initial string's form}, \\
\,\, \frac{\partial u}{\partial t}(x,0)=0, & \textrm{string starts from the rest},
\end{cases}
\end{equation*} 
and boundary conditions
\begin{equation*}
\begin{cases}
\,\, u(0,t)=0, & \textrm{string's left end fixed}, \\
\,\, u(1,t)=0, & \textrm{string's right end fixed}.
\end{cases}
\end{equation*}
Without loss of generality, we assume unitary the natural undeformed string's length. The solution of this problem approaches to a string's motion whose linear density is proportional to $(1+x)^{-2}$. The \emph{method of separation of variables} (i.e. $u(x,t)=X(x)T(t)$) gives the equations
\begin{equation}
X''+\frac{\lambda}{(1+x)^2}X=0,
\end{equation}
with boundary conditions $X(0)=X(1)=0$, and
\begin{equation}
T''+\lambda T=0,
\end{equation}
with initial conditions $T(0)=1$, $T'(0)=0$. In these equations, $\lambda$ is a constant parameter. \\
In (2), we are dealing with a Sturm-Liouville problem. To find out the eigenvalues, one searches the solution of (2) on the form $X(x)=(1+x)^a$, as we realize that (2) outcomes the associated characteristic equation
\begin{equation*}
a(a-1)+\lambda=0. \qquad \textrm{That is}, \qquad a=\frac{1}{2}(1\pm\sqrt{1-4\lambda}).
\end{equation*}
In order to satisfying $X(0)=0$, let us choose
\begin{equation*}
X(x)=(1+x)^{\frac{1}{2}(1+\sqrt{1-4\lambda})}-(1+x)^{\frac{1}{2}(1-\sqrt{1-4\lambda})}.
\end{equation*}
Thus, the boundary condition $X(1)=0$ becomes
\begin{equation*}
2^{\frac{1}{2}(1+\sqrt{1-4\lambda})}-2^{\frac{1}{2}(1-\sqrt{1-4\lambda})}=0, 
\quad \textrm{or}\quad 2^{\sqrt{1-4\lambda}}=1.
\end{equation*}
We next study all the possible cases for the eigenvalue $\lambda$ in the last above equation.
\begin{enumerate}
\item $\lambda<1/4$. Then $\sqrt{1-4\lambda}$ is real, and the equation does not have solution.
\item $\lambda=1/4$. Then the pair of solutions, above indicated, will not be independent. Indeed the functions $(1+x)^{1/2}$ and  $(1+x)^{1/2}\log(1+x)$ are linearly independent solutions of (2), in $(0,1)$. Nevertheless, although the last one satisfies the boundary condition at $x=0$, it does not vanish at $x=1$. Hence, $\lambda=1/4$ is not an eigenvalue.
\item $\lambda>1/4$. Then $\sqrt{1-4\lambda}$ is imaginary. We may even get two solutions by setting ($i=\sqrt{-1}$)
\begin{equation*}
Z(x)=X_1(x)+iX_2(x)=(1+x)^{\frac{1}{2}(1+i\sqrt{4\lambda-1})}=
(1+x)^{\frac{1}{2}}e^{\frac{1}{2}i\sqrt{4\lambda-1}\log(1+x)}
\end{equation*}
\begin{equation*}
=(1+x)^{\frac{1}{2}}\left\{\cos\left(\sqrt{\lambda-\frac{1}{4}}\log(1+x)\right)+
i\sin\left(\sqrt{\lambda-\frac{1}{4}}\log(1+x)\right)\right\},
\end{equation*}
being  the real and imaginary parts of $Z(x)$ two linearly independent solutions. For satisfying $X(0)=0$ one sets
\begin{equation*}
X(x)=(1+x)^{\frac{1}{2}}\sin\left(\sqrt{\lambda-\frac{1}{4}}\log(1+x)\right),
\end{equation*}
then the boundary condition $X(1)=0$ gives
\begin{equation*}
2^{\frac{1}{2}}\sin\left(\sqrt{\lambda-\frac{1}{4}}\log 2\right)=0.
\end{equation*}
Therefore, $\sqrt{\lambda-1/4}\log 2$ must be an integral multiple of $\pi$, i.e. $\sqrt{\lambda-1/4}\log 2=n\pi$, or
\begin{equation*}
\lambda\,\,\mapsto\,\, \lambda_n=\frac{n^2\pi^2}{\log^2 2}+\frac{1}{4}, \qquad n\in\mathbb{Z}^+.
\end{equation*} 
To these eigenvalues correspond the eigenfunctions
\begin{equation*}
X_n(x)=(1+x)^{\frac{1}{2}}\sin\left(n\pi\frac{\log(1+x)}{\log 2}\right),
\end{equation*}
\end{enumerate}
and these form a \emph{complete system}, whenever we impose to $f(x)$ certain conditions that we  shall see later. Moreover, $f(x)$ may be expanded in Fourier series
\begin{equation*}
f(x)\,\, \sim\,\, \sum_{n=1}^{\infty}c_n X_n(x),
\end{equation*}
where
\begin{equation*}
c_n=\frac{\int_0^1 f(x)(1+x)^{-\frac{3}{2}}\sin\left(n\pi\frac{\log(1+x)}{\log 2}\right)dx}
{\int_0^1 (1+x)^{-1}\sin^2\left(n\pi\frac{\log(1+x)}{\log 2}\right)dx}=
\frac{2}{\log 2}\int_0^1 f(x)(1+x)^{-\frac{3}{2}}\sin\left(n\pi\frac{\log(1+x)}{\log 2}\right)dx. 
\end{equation*}
The completeness above mentioned and the Fourier series converges absolutely and uniformly to $f(x)$ in $(0,1)$, only if $f(x)\in\mathcal{PC}^1(0,1)$, $f(0)=f(1)=0$ and $\int_0^1 f'\,^2(x) dx$ is \emph{finite}. {\footnote{A result due to Green-Parseval-Schwarz (GPS) and Bessel's inequality}.} \\
On the other hand, for satisfying (3) and its initial conditions, we choose the eigenfunction 
\begin{equation*}
T(t)\,\, \mapsto \,\, T_n(t)=\cos\sqrt{\lambda_n}\,t.
\end{equation*}
Thus, a solution of (1) is given by
\begin{equation*}
u_n(x,t)=X_n(x)T_n(t)=
(1+x)^{\frac{1}{2}}\sin\left(n\pi\frac{\log(1+x)}{\log 2}\right)\cos\sqrt{\lambda_n}\,t,
\end{equation*}
and the general solution of (1) may be determined as a linear (infinite) combination of these eigenfunctions, that is
\begin{equation*}
u(x,t)\,\, \sim\,\, \sum_{n=1}^{\infty}c_n u_n(x,t)=\sum_{n=1}^{\infty}c_n X_n(x)T_n(t).
\end{equation*}
So that, the string's problem (1) has the formal solution
\begin{equation}
u(x,t)=\sum_{n=1}^{\infty}c_n (1+x)^{\frac{1}{2}}\sin\left(n\pi\frac{\log(1+x)}{\log 2}\right)
\cos\left(\sqrt{\frac{n^2\pi^2}{\log^2 2}+\frac{1}{4}}\;t\right).
\end{equation}
This series converges uniformly, and hence satisfies the initial and boundary conditions, as the series for $f(x)$ converges uniformly. However, in order to assure continuous derivatives and the partial differential equation (1) to be satisfied, we need suppose that the series for $f''(x)$ converges uniformly, i.e. we must suppose that $f(x)$ to be \emph{regular} {\footnote{i.e. $f(x)\in \mathcal{C}^2(0,1)$.}}, that $f(0)=f(1)=f''(0)=f''(1)=0$, and that $\int_0^1 f'''\,^2(x) dx$ to be finite. {\footnote{By GPS, again}.}




 
 
%%%%%
%%%%%
\end{document}
