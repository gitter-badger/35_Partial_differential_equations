\documentclass[12pt]{article}
\usepackage{pmmeta}
\pmcanonicalname{ExampleOfSolvingTheHeatEquation}
\pmcreated{2014-09-28 17:02:21}
\pmmodified{2014-09-28 17:02:21}
\pmowner{pahio}{2872}
\pmmodifier{pahio}{2872}
\pmtitle{example of solving the heat equation}
\pmrecord{25}{37647}
\pmprivacy{1}
\pmauthor{pahio}{2872}
\pmtype{Example}
\pmcomment{trigger rebuild}
\pmclassification{msc}{35Q99}
\pmsynonym{stationary example of heat equation}{ExampleOfSolvingTheHeatEquation}
%\pmkeywords{time-independent}
\pmrelated{LaplacesEquation}
\pmrelated{BlackScholesPDE}
\pmrelated{AnalyticSolutionOfBlackScholesPDE}
\pmrelated{SolvingTheWaveEquationByDBernoulli}
\pmrelated{TimeDependentExampleOfHeatEquation}
\pmrelated{ExampleOfSummationByParts}

% this is the default PlanetMath preamble.  as your knowledge
% of TeX increases, you will probably want to edit this, but
% it should be fine as is for beginners.

% almost certainly you want these
\usepackage{amssymb}
\usepackage{amsmath}
\usepackage{amsfonts}

% used for TeXing text within eps files
%\usepackage{psfrag}
% need this for including graphics (\includegraphics)
\usepackage{graphicx}
% for neatly defining theorems and propositions
 \usepackage{amsthm}
% making logically defined graphics
%%%\usepackage{xypic}

% there are many more packages, add them here as you need them

% define commands here

\theoremstyle{definition}
\newtheorem*{thmplain}{Theorem}
\begin{document}
Let a \PMlinkescapetext{thin square-formed plate of heat conducting homogeneous material be in the $xy$-plane with sides on the $x$-axis (isolated), on the line\, $y = \pi$ (held at the constant temperature\, $u = C$), and on the vertical lines\, $x = 0$\, and\, $x = \pi$ (both held at the constant temperature\, $u = 0$)}.\, Determine the temperature function\, $(x,\,y)\mapsto u(x,\,y)$\, on the plate, when the faces of the plate are \PMlinkescapetext{isolated}.\\

The \PMlinkname{equation of the heat flow}{HeatEquation} in this \PMlinkescapetext{stationary} case is
\begin{align}
\nabla^2 u \;\equiv\; u''_{xx}+u''_{yy} \;=\; 0
\end{align}
under the boundary conditions
$$u(0,\,y) = 0, \qquad u(\pi,\,y) = 0, \qquad u(x,\,\pi) = C, 
\qquad u'_y(x,\,0) = 0.$$
We first try to separate the variables, i.e. seek the solution of (1) of the form
$$u(x,\,y) \;:=\; X(x)\,Y(y).$$
Then we get
$$u'_x = X'Y, \qquad u''_{xx} = X''Y, \qquad u'_y = XY', 
\qquad u''_{yy} = XY'',$$
and thus (1) gets the form
\begin{align}
X''Y+XY'' \;=\; 0
\end{align}
and the boundary conditions
$$X(0) \;=\; X(\pi) \;=\; 0, \quad X(x) \;=\; \frac{C}{Y(\pi)}, \quad Y'(0) \;=\; 0.$$
We separate the variables in (2):
$$\frac{X''}{X} \;=\; -\frac{Y''}{Y}$$
This equation is not possible unless both sides are equal to a same negative \PMlinkescapetext{constant} $-k^2$, which implies for\, $X'' = -k^2X$\, the solution
$$X \;:=\; C_1\cos{kx}+C_2\sin{kx}$$
and for\, $Y'' \;=\; k^2Y$\, the solution
$$Y \;:=\; D_1\cosh{ky}+D_2\sinh{ky}.$$
The two first boundary conditions give\, $0 = X(0) = C_1$,\, $0 = X(\pi) = 0+C_2\sin{k\pi}$,\, and since\, $C_2 \ne 0$,\, we must have\, $\sin{k\pi} = 0$,\, i.e.
$$0 \;<\; k \;:=\; n \;=\; 1,\,2,\,3,\,\ldots$$
Therefore 
$$X(x) \;:=\; C_2\sin{nx}, \quad Y'(y) \;\equiv\; nD_1\sinh{ny}+nD_2\cosh{ny}.$$
The fourth boundary condition now yields that\, $0 = Y'(0) = nD_2$;\,
thus\, $D_2 = 0$\, and\, $Y(y) := D_1\cosh{ny}.$\, So (1) has infinitely many solutions
\begin{align}
u_n \;:=\; C_2D_1\sin{nx}\cosh{ny} \;=\; A_n\sin{nx}\cosh{ny}
\end{align}
with\, $n\in\mathbb{Z}_+$\, and they all satisfy the boundary conditions except the third.\, Because of the linearity of (1), also the sum 
$$u \;:=\; \sum_{n=1}^\infty A_n\sin{nx}\cosh{ny}$$
of the functions (3) satisfy (1) and those boundary conditions, provided that this series converges.\, The third boundary condition requires that
$$C \;=\; u(x,\,\pi) \;=\; \sum_{n=1}^\infty A_n\sin{nx}\cosh{n\pi} \;=\;
    \sum_{n=1}^\infty(A_n\cosh{n\pi})\sinh{nx}$$
on the interval\, $0 \leqq x \leqq \pi$.\, But this is the Fourier sine series of the constant function\, $x \mapsto C$\, on the half-interval\, $[0,\,\pi]$,\, whence 
$$A_n\cosh{n\pi} \;=\; \frac{2}{\pi}\int_0^\pi C\sin{nx}\,dx \;=\; 
  \frac{2C}{n\pi}(1\!-\!(-1)^n) \quad \forall n\in\mathbb{Z}_+.$$
The \PMlinkname{even}{EvenNumber} $n$'s here give 0 and the \PMlinkname{odd}{EvenNumber} give
$$A_{2m+1} \;:=\; \frac{4C}{(2m\!+\!1)\pi\cosh(2m\!+\!1)\pi}
\quad (m = 0,\,1,\,2,\,\ldots)$$

Thus we obtain the solution
$$u(x,\,y) \;:=\; 
\frac{4C}{\pi}\sum_{m=0}^\infty\frac{\sin(2m\!+\!1)x\cosh(2m\!+\!1)y}
                                    {(2m\!+\!1)\cosh(2m\!+\!1)\pi}.$$
It can be shown that this series converges in the whole \PMlinkescapetext{square} of the plate.

\subsection*{Visualization of the solution}

\begin{figure}[!htb]
\begin{center}
\includegraphics{heat-surface.eps}
\caption{Surface plot of the solution $u(x,\,y)$, for $C = 1$} 
\end{center}
\end{figure}

\begin{figure}[!htb]
\begin{center}
\includegraphics{heat-color.eps}
\caption{Color-coded plot of the temperature $u(x,\,y)$}
\end{center}
\end{figure}

\textbf{Remark.}\, The function $u$ has been approximated in the plot by computing a partial sum of the true infinite-series solution.\, However,
there is substantial numerical error in the approximate solution near\, 
$y = \pi$,\, evident in the small oscillations observed in the surface plot, that should not be there in \PMlinkescapetext{theory}.\, This phenomenon is actually inevitable given that the boundary conditions are actually discontinuous at the corners\, $(0,\,\pi)$\, and\, $(\pi,\,\pi)$.

More precisely, observe that when\, $y = \pi$,\, the \PMlinkescapetext{formula} for\, $u(x,\,y)$\, reduces to the Fourier series
\[
\frac{4C}{\pi} \left( \sin{x}+\frac{\sin{3x}}{3}+\frac{\sin{5x}}{5}+\dotsb \right)
\]
for the discontinuous function on\, $[-\pi,\,\pi]$:
\[
x \mapsto \begin{cases}
 C\,, & 0 < x < \pi \\
 -C\,, & -\pi < x < 0
\end{cases}\,
\]
That means the Fourier \PMlinkescapetext{expansion} will necessarily be subject to the {\em Gibbs phenomenon}.\, Of course, the series also cannot converge absolutely; in other \PMlinkescapetext{words, the terms} of the series decay too slowly in magnitude, adversely affecting the numerical solution.

\begin{itemize}
\item
\PMlinkexternal{Python program to compute\, $u(x,\,y)$\, and produce the two figures}{http://gold-saucer.afraid.org/math/planetmath/ExampleOfSolvingTheHeatEquation/heat.py}
\end{itemize}

%%%%%
%%%%%
\end{document}
