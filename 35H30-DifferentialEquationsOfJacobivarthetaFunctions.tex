\documentclass[12pt]{article}
\usepackage{pmmeta}
\pmcanonicalname{DifferentialEquationsOfJacobivarthetaFunctions}
\pmcreated{2013-03-22 14:41:19}
\pmmodified{2013-03-22 14:41:19}
\pmowner{rspuzio}{6075}
\pmmodifier{rspuzio}{6075}
\pmtitle{differential equations of Jacobi $\vartheta$ functions}
\pmrecord{10}{36296}
\pmprivacy{1}
\pmauthor{rspuzio}{6075}
\pmtype{Theorem}
\pmcomment{trigger rebuild}
\pmclassification{msc}{35H30}

% this is the default PlanetMath preamble.  as your knowledge
% of TeX increases, you will probably want to edit this, but
% it should be fine as is for beginners.

% almost certainly you want these
\usepackage{amssymb}
\usepackage{amsmath}
\usepackage{amsfonts}

% used for TeXing text within eps files
%\usepackage{psfrag}
% need this for including graphics (\includegraphics)
%\usepackage{graphicx}
% for neatly defining theorems and propositions
%\usepackage{amsthm}
% making logically defined graphics
%%%\usepackage{xypic}

% there are many more packages, add them here as you need them

% define commands here
\begin{document}
The theta functions \PMlinkescapetext{satisfy} the following partial differential equation:
 $${\pi i \over 4} {\partial^2 \vartheta_i \over \partial z^2} + {\partial \vartheta_i \over \partial \tau} = 0$$

It is easy to check that each \PMlinkescapetext{term} in the series which define the theta functions \PMlinkescapetext{satisfies} this differential equation.  Furthermore, by the Weierstrass M-test, the series obtained by differentiating the series which define the theta functions term-by-term converge absolutely, and hence one may compute derivatives of the theta functions by taking derivatives of the series term-by-term.

Students of mathematical physics will recognize this equation as a one-dimensional diffusion equation.  Furthermore, as may be seen by examining the series defining the theta functions, the theta functions approach periodic delta distributions in the limit $\tau \to 0$.  Hence, the theta functions are the Green's functions of the one-dimensional diffusion equation with periodic boundary conditions.
%%%%%
%%%%%
\end{document}
