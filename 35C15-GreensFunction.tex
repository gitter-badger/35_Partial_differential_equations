\documentclass[12pt]{article}
\usepackage{pmmeta}
\pmcanonicalname{GreensFunction}
\pmcreated{2013-03-22 14:43:36}
\pmmodified{2013-03-22 14:43:36}
\pmowner{PrimeFan}{13766}
\pmmodifier{PrimeFan}{13766}
\pmtitle{Green's function}
\pmrecord{7}{36355}
\pmprivacy{1}
\pmauthor{PrimeFan}{13766}
\pmtype{Definition}
\pmcomment{trigger rebuild}
\pmclassification{msc}{35C15}
\pmrelated{PoissonsEquation}

% this is the default PlanetMath preamble.  as your knowledge
% of TeX increases, you will probably want to edit this, but
% it should be fine as is for beginners.

% almost certainly you want these
\usepackage{amssymb}
\usepackage{amsmath}
\usepackage{amsfonts,euscript}

% used for TeXing text within eps files
%\usepackage{psfrag}
% need this for including graphics (\includegraphics)
%\usepackage{graphicx}
% for neatly defining theorems and propositions
%\usepackage{amsthm}
% making logically defined graphics
%%%\usepackage{xypic}

% there are many more packages, add them here as you need them

% define commands here
%%%%% My special things
\newcommand{\DSp}{(\EuScript{F}(\Omega))^*}
\newcommand{\FOx}{\EuScript{F}(\Omega_x)}
\newcommand{\GOy}{\EuScript{G}(\Omega_y)}
%%%%%%%%%%%%%%%%%%%%%%%
\begin{document}
\section*{Some general preliminary considerations}
Let $(\Omega,\mu)$ be a bounded measure space and $\EuScript{F}(\Omega)$ be a linear function
space of bounded functions defined on $\Omega$, i.e. $\EuScript{F}(\Omega)\subset\EuScript{L}^\infty(\Omega)$.
We would like to note two types of functionals from the dual space $\DSp$, which
will be used here:
\begin{enumerate}
\item Each function $g(x)\in\EuScript{L}^1(\Omega)$ defines a functional $\varphi\in\DSp$ in the
    following way:
    $$
        \varphi(f)=\int\limits_{\Omega} g(x)\,f(x)\,d\mu.
    $$
    Such functional we will call \textit{regular} functional and function $g$ --- its \textit{generator}.

\item For each $x\in\Omega$, we will consider a functional $\delta_x\in\DSp$ defined as follows:
    \begin{equation}\label{dFn}
        \delta_x(f)=f(x).
    \end{equation}
    Since generally, we can not speak about values at the point for functions from $\EuScript(L)^\infty$,
    in the following, we assume some regularity for functions from considered spaces, so that
    (\ref{dFn}) is correctly defined.

\end{enumerate}


\section*{Necessary notations and motivation}
Let $(\Omega_x,\mu_x),\,(\Omega_y,\mu_y)$ be some bounded measure spaces; $\FOx,\GOy$ be some
linear function spaces. Let $A:\FOx\rightarrow\GOy$ be a linear operator which has a well-defined
inverse $A^{-1}:\GOy\rightarrow\FOx$.

Consider an operator equation:
\begin{equation}\label{OpEq}
    Af=g
\end{equation}
where $f\in\FOx$ is unknown and $g\in\GOy$ is given. We are interested to have an integral representation
for solution of (\ref{OpEq}). For this purpose we write:
$$
    f(x)=\delta_x(f)=\delta_x(A^{-1}(g))=[\, (A^{-1})^*\delta_x \,](g).
$$

\section*{Definition of Green's function}
If $\forall x\in\Omega_x$ the functional $(A^{-1})^*\delta_x$ is regular with generator
$G(\cdot,y)\in\EuScript{L}^1(\Omega_y)$, then $G$ is called \textbf{Green's function of
operator $A$} and solution of (\ref{OpEq}) admits the following integral representation:
$$
    f(x)=\int\limits_{\Omega_y}G(x,y)\,g(y)\,d\mu_y
$$
%%%%%
%%%%%
\end{document}
